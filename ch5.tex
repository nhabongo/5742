\chapter{Conclusões}
\label{ch:5}

Por mais que o projeto RAMCloud ja tenha uma versão que os desenvolvedores dizem ser utilizável, algumas limitações, deficiencias e a falta de uma aplicação que
realmente possa tirar vantagem dessa arquitetura podem acabar retardando ou mesmo inviabilizando sua utilização atualmente. O sistema em geral depende fortemente
do desempenho da rede de intercomunicação, nada menos do que a técnologia de ponta que existe hoje, o que pode implicar em alto custo de implantação e eventuais
falhas de hardware, software ou mesmo de projeto dessas tecnologias. Além disso, para que o RAMCloud funcione como previsto, todos os servidores precisam estar
no mesmo datacenter uma vez que está implicito que os servidores podem se comunicar com baixa latência. 
\par
Do ponto de vista de armazenamento, para permitir que o sistema se recupere de falhas em menos de dois segundos, cada servidor de backup não pode armazenar mais 
do que cerca de 500MB de log, ou seja, em um cluster com 10 nós cada servidor master pode armazenar no máximo 5GB de DRAM. Além disso, o modelo de dados chave-valor adotado é bem simples e eficiente para acesso, mas aplicações que necessitam de um modelo de dados mais estruturado, com índices secundários e transações de multiplos objetos (ou mesmo um modelo ACID) teriam que ser totalmente reescritas para funcionar no RAMCloud. Falta também algum sistema de proteção de acesso aos dados, atualmente
qualquer cliente pode modificar qualquer dado armazenado.
\par
Talvez ainda não exista um nicho específico para o projeto RAMCloud embora não deixe de ser um projeto interessante e bastante promissor, basta analisar o mercado de software para chegar a conclusão de que existe uma demanda cada vez maior, por parte das aplicações web, para manter dados em memória RAM. Técnologias oriundas da memcomputação, principalmente os memristores, podem viabilizar a criação de um novo tipo de memória que agregue baixa latência, velocidade superior as DRAM e capacidade de um disco rigido, tornando não só o RAMCloud como diversas técnologias atuais obsoletas.

\vfill
\begin{flushright}
\textit{One machine can do the work of fifty ordinary men. \\
No machine can do the work of one extraordinary man.}\\
--Elbert Hubbard

\end{flushright}
